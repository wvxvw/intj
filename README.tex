% Created 2013-10-02 Wed 09:57
\documentclass[11pt]{article}
\usepackage[utf8]{inputenc}
\usepackage[T1]{fontenc}
\usepackage{fixltx2e}
\usepackage{graphicx}
\usepackage{longtable}
\usepackage{float}
\usepackage{wrapfig}
\usepackage{soul}
\usepackage{textcomp}
\usepackage{marvosym}
\usepackage{wasysym}
\usepackage{latexsym}
\usepackage{amssymb}
\usepackage{hyperref}
\tolerance=1000
\usepackage{amsmath}
\providecommand{\alert}[1]{\textbf{#1}}

\title{README}
\author{Oleg Sivokon}
\date{\today}
\hypersetup{
  pdfkeywords={},
  pdfsubject={},
  pdfcreator={Emacs Org-mode version 7.9.3f}}

\begin{document}

\maketitle

\setcounter{tocdepth}{3}
\tableofcontents
\vspace*{1cm}

\section{Original Text of the Assignment}
\label{sec-1}
\subsection{\texttt{Walla!} Backend Engineers Test Assignment}
\label{sec-1-1}

   
   Heya,
   In order to evaluate your skills as a backend Web engineer, we
   would like you to take this short development assignment, at your
   own free time.
   
\subsubsection{General Guidelines:}
\label{sec-1-1-1}

\begin{enumerate}
\item Please provide
\begin{itemize}
\item working code.
\item a brief design document explaining what you did.
\item a link to a server where your code can be tested.
\end{itemize}
\item Feel free to search the Web for inspiration, but please do not
       copy code from anywhere, keep it 100\% yours.
\item Use Python as your language of implementation. Micro​ Frameworks
       and utility libraries are allowed, but not a full blown MVC.
\end{enumerate}
\subsubsection{The Test Assignment:}
\label{sec-1-1-2}

    You are to design and implement the API and server side modules
    for a simplified twitter like social updates site, and a simple
    AJAX client for it. The application contains users; Each user can
    post short text messages, follow other users, and get a feed of
    the latest updates from the users he is following. You can also
    get a global feed for all the users.
    
\begin{enumerate}
\item Implement an HTTP based (ReSTful in the loose sense of the
       term) API that exposes the following calls (no need for
       authentication, choose the format you like):
\begin{itemize}
\item \texttt{CreateUser(UserName)}
\item \texttt{PostMessage(UserId, MessageText)}
\item \texttt{Follow(FollowingUser, FollowedUser)}
\item \texttt{Unfollow(FollwingUser, UnfollowedUser)}
\item \texttt{GetFeed(ForUserId)} \emph{Note the aggregate feed of the users}
         \emph{this user is following!}
\item \texttt{GetGlobalFeed()}
\end{itemize}
\item Please create an ajax based HTML page for this API:
\begin{itemize}
\item Interface ​
         It should support getting the messages from the followed
         users per user Id, with real time updates, and the ability to
         post messages. You do not need to implement
         following/unfollowing. No styling required.
\item Implementation
         Use of toolkits (jQuery, Zepto, bootstrap..) is fine, but no
         frameworks plz (Angular, Backbone..). No need for mobile
         adaption. Strive for efficiency and readability.
\end{itemize}
\end{enumerate}
         
    Good Luck!
    The Walla! Backend Engineering Team
\section{What Needs to Be Done}
\label{sec-2}
\subsection{\textbf{DONE} Install and test Tornado \textbf{:setup:}}
\label{sec-2-1}

   \texttt{DEADLINE:} \textit{2013-10-01 Tue}

\begin{itemize}
\item State ``DONE''       from ``TODO''       \textit{2013-10-02 Wed 08:20} \\
This was the easy part.
\end{itemize}
   Was suggested during the interview
\subsection{\textbf{CANCELED} Make sure MySQL binding is up-to-date \textbf{:setup:alternatives:}}
\label{sec-2-2}

   \texttt{DEADLINE:} \textit{2013-10-01 Tue}

\begin{itemize}
\item State ``CANCELED''   from ``TODO''       \textit{2013-10-02 Wed 08:21} \\
I'll be using \texttt{neo4j}, that's decided.
\end{itemize}
   I have some vague memories of it being problematic to install,
   perhaps it required \texttt{mysql-devel} package? Can \texttt{setuptools}
   install RPM / DEB?
\subsection{\textbf{DONE} Maybe try \texttt{py2neo} \textbf{:setup:alternatives:}}
\label{sec-2-3}

   \texttt{DEADLINE:} \textit{2013-10-02 Wed}

\begin{itemize}
\item State ``DONE''       from ``TODO''       \textit{2013-10-02 Wed 08:21} \\
Yup.
\end{itemize}
   Just for the sake of experiment? There are couple of videos to
   watch. If I can get it to run easily, I'd probably use that.
\subsection{\textbf{TODO} Figure out what Apache httpd module to use \textbf{:setup:info:}}
\label{sec-2-4}

   \texttt{DEADLINE:} \textit{2013-10-01 Tue}

   There is some controversy about \texttt{WSGI} / \texttt{FSGI} / \texttt{mode\_python}
   Need to make sure I'm not using some outdated / unsupported
   library.
\subsection{\textbf{TODO} Brush up on Tweeter Bootstrap \textbf{:setup:info:}}
\label{sec-2-5}

   \texttt{DEADLINE:} \textit{2013-10-01 Tue}

   This is most likely candidate for the GUI. Need to see what
   components are already there, and what's missing (unlikely).
\subsection{\textbf{TODO} Find out what most people use for unit testing \textbf{:setup:info:}}
\label{sec-2-6}

   \texttt{DEADLINE:} \textit{2013-10-02 Wed}

   We'll see what Stackoverflow has to say\ldots{}
   
\subsection{\textbf{TODO} Ropemacs was giving me headache lately \textbf{:setup:}}
\label{sec-2-7}

   \texttt{DEADLINE:} \textit{2013-10-02 Wed}

   See if the stupid thing still works. Maybe we shall do w/o it.
   There was some other package\ldots{} Probably elpy or some such,
   maybe it makes sense to check it out.
\subsection{\textbf{DONE} Start a GitHub repo (and fix the setup) \textbf{:setup:}}
\label{sec-2-8}

   \texttt{DEADLINE:} \textit{2013-10-02 Wed}

\begin{itemize}
\item State ``DONE''       from ``TODO''       \textit{2013-10-02 Wed 08:21} \\
Looks like it also can now display Org files markup, so no need to
     export to Markdown
\end{itemize}
   Not today.
\section{Project Description}
\label{sec-3}

  Right now I'm thinking about making it into a \href{http://en.wikipedia.org/wiki/Nomic}{nomic} game.  The basic
  idea would be that by posting users may, in general, score
  points. More so, if someone comments on their posts, or if they
  themselves comment. But it would be also possible to loose points
  due to bad posts or comments. Sort of karma, just like there is one
  on the StackExchange sites.

  I'm definitely going with \texttt{py2neo}, especially because it uses
  \texttt{tornado}. It's not as efficient as SQL (at the moment), but who
  cares?
\subsection{Nodes and relations I will need}
\label{sec-3-1}

   
   Nodes
\begin{itemize}
\item \texttt{User\#\{ "id": int, "name": string, "password": string \}}
\item \texttt{Message\#\{ "id": int, "text" : string, "score": int \}}
\end{itemize}
   
   Relations
\begin{itemize}
\item $user \; \boldsymbol{Follows} \; user$
\item $user \; \boldsymbol{FollowedBy} \; user$
\item $message \; \boldsymbol{PostedBy} \; user$
\item $user \; \boldsymbol{AuthorOf} \; message$
\item $user \; \boldsymbol{Rates(points)} \; message$
\end{itemize}

\end{document}
